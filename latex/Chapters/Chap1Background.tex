
% Chapter Template

\chapter{Background}
\label{chap:Background}

\section{Notation}

A \textbf{sample} $\vec z$ is an observation drawn from a joint distribution over the domain $\set Z = \set X \times \set Y$. 
It consists of a tuple $(\vec x, y)$ of an \textbf{input} $\vec x$ and a \textbf{label} $y$. 

\noindent
The input space $\set X$ typically is $\R^d$ and $\set Y$, the set of possible labels, is a set of integers $\{1, \dots, C\}$ for some whole number $C$, also known as the \textbf{number of classes}.

\noindent
A \textbf{bracket-notation} $\finiteset{\, \cdot \,}$ is used to denote the set of all \textit{finite non-empty} subsets.
$\set A = \{\vec z^{(i)}\}_{i=1}^{| \set A |} = \{(\vec x^{(i)}, y^{(i)})\}_{i=1}^{| \set A |} \in \finiteset{\set Z}$ is called a \textbf{data set}. $\set A $ contains \textit{i.i.d.} (independent and identically distributed) samples drawn from some distribution over $\set Z$.
A \textbf{sample batch} or simply \textbf{batch} is a \textit{non-empty} subset thereof.

\noindent
$\mean$ and $\var$ are used to denote the empirical sample mean and variance of the input.
\begin{align}
    \mean ({\set A}) &= \frac 1 {\lvert \set A \rvert} \sum _{i=1}^{\lvert \set A \rvert} \vec x^{(i)} \; \in \set X
    \label{eqn:sample_mean} \\
    \var ({\set A}) &= \frac 1 {\lvert \set A \rvert} \sum _{i=1}^{\lvert \set A \rvert} (\vec x^{(i)} - \mean ({\set A}))^2 \; \in \set X 
    \label{eqn:sample_var} 
\end{align}


Given a sample batch $\set A$, the set $\set A|_c$ is the subset of $\set A$ \textbf{constrained} to samples of label $c$.
\begin{equation}
    \set A|_c = \{(\vec x, y) \in \set A \mid y = c\}
    \label{eqn:class_constrained}
\end{equation}

A \textbf{feature-mapping} is any function from input space $\R^d$ to $\R^m$ for some $m$.
A data set to data set mapping is obtained by applying a feature-mapping $\varphi$ to every input. Given a data set $\set A$, and a feature-mapping $\varphi$:
\begin{equation}
    \varphi(\set A) := \{(\varphi(\vec x), y) \mid (\vec x, y) \in \set A\} \;
    \in \finiteset Z
    \label{eqn:feature_mapping}
\end{equation}



\section{Loss function}
Given two data sets $\set A, \set B$, a \textbf{loss function} or \textbf{objective function} is a function $\loss : \finiteset{\set Z} \times \finiteset{Z} \to \R$ that is \textit{piecewise continuously differentiable} with regard to its inputs.
I. e. where 
\[
    \loss (\set A, \set B) = \loss (
    \,\{(\vec x^{(i)}_{\set A}, y^{(i)}_{\set A})\}_{i=1}^{|\set A|}\,,
    \,\{(\vec x^{(j)}_{\set B}, y^{(j)}_{\set B})\}_{j=1}^{|\set B|}\,) 
\]
is piecewise $\mathcal C^1$ with regard to $\vec x^{(i)}_{\set A}$
for all $i=1, \dots, |\set A|$, and $\vec x^{(j)}_{\set B}$ for all $j=1, \dots, |\set B|$.
Note that, for the variance to behave as expected, the batch should contain at least two samples.

For a given feature-mapping $\varphi$ that is piecewise $\mathcal C ^1$, the \textbf{statistics-loss} or simply loss, is defined as follows.
% 
\begin{equation}
    \loss _{\varphi} (\set A, \set B) = 
    \|\mean ({\varphi (\set A)}) - \mean ({\varphi (\set B)})\| +
    \|\var ({\varphi (\set A)}) - \var ({\varphi (\set B)})\|
    \label{eqn:statistics_loss}
\end{equation}
% 
A \textbf{class-dependent} loss can be obtained as follows:
\begin{equation}
\label{eqn:class_statistics_loss}
    \loss _\varphi ^{\mathcal C} (\set A, \set B) =
    \sum _{\substack{c = 1 \dots C \\ \set A|_c,\, \set B|_c \text{ contain} \\ \text{at least 2 samples}}}
    \begin{alignedat}[t]{1}
        \|\mean ({\varphi (\set A|_c)}) - \mean({\varphi (\set B|_c)}) &\| \\
        {} + \|\var ({\varphi (\set A|_c)}) - \var ({\varphi (\set B|_c)}) &\| 
    \end{alignedat}
\end{equation}
% \begin{equation}
%     \loss _\varphi ^{\mathcal C} (\set A, \set B) =
%     \sum _{\substack{c = 1 \dots C \\ \set A|_c,\, \set B|_c \text{ contain} \\ \text{at least 1 sample}}} 
%     \substack{{} + \|\var ({\varphi (\set A|_c)}) - \var ({\varphi (\set B|_c)})\|}{\|\mean ({\varphi (\set A|_c)}) - \mean({\varphi (\set B|_c)})\|}
%     \label{eqn:class_statistics_loss}
% \end{equation}

\subsection{Combinations}
% 
This notion can be extended to a collection of feature-maps [$\varphi_1, \dots, \varphi_n$].
A \textbf{collection} is used to refer to a \textit{non-empty finite} set.
\begin{align*}
    \loss_{[\varphi_1, \dots, \varphi_n]} (\set A, \set B) &= 
    \sum_{i=1}^n \loss_{\varphi_i} (\set A, \set B) \\
    %
    \loss_{[\varphi_1, \dots, \varphi_n]}^{\mathcal C} (\set A, \set B) &= 
    \sum_{i=1}^n \loss_{\varphi_i}^{\mathcal C} (\set A, \set B)
\end{align*}
