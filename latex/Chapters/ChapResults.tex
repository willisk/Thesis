\section{Results}

\subsection{Quantitative Results}

\subsubsection{GMM}

\begin{table}
\centering
\footnotesize
\pgfplotstabletypeset[
gmm,
display columns/0/.style={column name=\textbf{Baseline}, column type=l, string type},
]{figures/reconstruction_GMM_baseline.csv}
\caption{baseline scores for comparison}
\label{tab:gmmbaseline}
\end{table}

\begin{table}
\centering
\footnotesize
\pgfplotstabletypeset[
gmm,
every row 6 column 1/.style={highlight},
every row 8 column 1/.style={highlight},
every row 10 column 1/.style={highlight},
every row 11 column 1/.style={highlight},
every row 7 column 2/.style={highlight},
every row 4 column 3/.style={highlight},
every row 8 column 4/.style={highlight},
every row 8 column 5/.style={highlight},
]{figures/reconstruction_GMM_results.csv}
\caption{Metrics on reconstruction results after 100 optimization epochs for Gaussian Mixture Model data set}
\label{tab:gmmresults}
\end{table}

\begin{table}
\centering
\footnotesize
\pgfplotstabletypeset[
gmm,
every row 1 column 1/.style={highlight},
every row 7 column 2/.style={highlight},
every row 7 column 3/.style={highlight},
every row 4 column 4/.style={highlight},
every row 7 column 5/.style={highlight},
]{figures/reconstruction_GMM_results_raw.csv}
\caption{Metrics on reconstruction results after 100 optimization epochs for Gaussian Mixture Model data set}
\label{tab:gmmresultsraw}
\end{table}

All proposed methods perform reasonably well. 
Although, it is important that the loss function used in the optimization process
is a composite loss from the proposed loss between the statistics
and the original cross-entropy loss used in the optimization of the neural network.
\[
    r_{stats} \loss(\set A, \set B) + r_{crit}\loss_{crit}(\set B, y_{\set B})
\]
Since the original criterion loss alone already performs quite good, the proposed losses act more
A comparison of the results for the case $r_{crit}=0$ can be found in table \ref{tab:gmmresultsraw}



\subsubsection{MNIST}
\begin{table}
\label{tab:mnistbaseline}
\centering
\footnotesize
\pgfplotstabletypeset[
images,
display columns/0/.style={column name=\textbf{Baseline}, column type=l, string type},
]{figures/reconstruction_MNIST_baseline.csv}
\caption{baseline scores for comparison}
\end{table}

\begin{table}
\label{tab:mnistresults}
\centering
\footnotesize
\pgfplotstabletypeset[
images,
every row 11 column 1/.style={highlight},
every row 11 column 2/.style={highlight},
every row 11 column 3/.style={highlight},
every row 11 column 4/.style={highlight},
every row 5 column 5/.style={highlight},
every row 6 column 6/.style={highlight},
]{figures/reconstruction_MNIST_results.csv}
\caption{Metrics on reconstruction results after 100 optimization epochs for MNIST data set}
\end{table}


\subsubsection{CIFAR10}
\begin{table}
\label{tab:cifar10baseline}
\centering
\footnotesize
\pgfplotstabletypeset[
images,
display columns/0/.style={column name=\textbf{Baseline}, column type=l, string type},
]{figures/reconstruction_CIFAR10_baseline.csv}
\caption{baseline scores for comparison}
\end{table}

\begin{table}
\label{tab:cifar10results}
\centering
\footnotesize
\pgfplotstabletypeset[
images,
every row 2 column 1/.style={highlight},
every row 4 column 1/.style={highlight},
every row 3 column 2/.style={highlight},
every row 4 column 3/.style={highlight},
every row 3 column 4/.style={highlight},
every row 3 column 5/.style={highlight},
every row 3 column 6/.style={highlight},
]{figures/reconstruction_CIFAR10_results.csv}
\caption{Metrics on reconstruction results after 100 optimization epochs for CIFAR10}
\end{table}






















