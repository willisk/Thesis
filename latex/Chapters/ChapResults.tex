\section{Results}

\subsection{Quantitative Results}

\pgfplotstableset{
    highlight/.style={postproc cell content/.append style={/pgfplots/table/@cell content/.add={$\bf}{$}}},
}

% postproc cell content/.append style={/pgfplots/table/@cell content/.add={$\noexpand\bf}{$}},

\pgfplotstableset{
col sep = comma,
string replace*={_}{\textsubscript},
every head row/.style={before row=\toprule,after row=\midrule},
every last row/.style={after row=\bottomrule},
every column/.style={column type=l, precision=1, zerofill},
columns={[index]0, [index]1, [index]2, [index]3, [index]4, [index]5, [index]6},
display columns/0/.style={column name=\textbf{Method}, column type=l, string type},
display columns/1/.style={column name=\textbf{Accuracy} [\%], multiply with=100},
display columns/2/.style={column name=\makecell{validation \\ \textbf{Accuracy}} [\%], column type=l, multiply with=100},
display columns/3/.style={column name=\makecell{verifier \\ \textbf{Accuracy}} [\%], column type=l, multiply with=100},
display columns/4/.style={column name=\textbf{l2-error}},
display columns/5/.style={column name=\textbf{PSNR}},
display columns/6/.style={column name=\textbf{SSIM} [\%], column type=l, multiply with=100},
}
\pgfplotstabletypeset{figures/reconstruction_CIFAR10_baseline.csv}
\pgfplotstabletypeset[
every row 2 column 1/.style={highlight},
every row 4 column 1/.style={highlight},
every row 3 column 2/.style={highlight},
every row 4 column 3/.style={highlight},
every row 3 column 4/.style={highlight},
every row 3 column 5/.style={highlight},
every row 3 column 6/.style={highlight},
]{figures/reconstruction_CIFAR10_results.csv}






















