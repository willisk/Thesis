
\chapter{Experiments}
\label{chap:Experiments} 

The success of the reconstruction largely depends on the representation of the data
after the feature-map is applied.
Relying on the mean and variance of the data to accurately describe the data distribution 
comes with a strong assumption.
More sophisticated ways of capturing data distributions exist, yet mean and variance were chosen due to their simplicity in evaluation and computing of the gradients.

The feature-map ideally is to "disentangle" the data space to enable 
these simple statistics to capture enough of the complexity of the data set.

In order to assess the efficacy of the neural network's latent representations 
and their ability to capture the data distribution by these statistics,
several different mappings methods will be compared. 

First, in order to establish the notation used throughout, in subsection \ref{sec:nn_def}, 
a short introduction on the multi-layer perceptron (MLP) is given,
Then, the various feature-mappings or methods will be presented in \ref{sec:methods}.
Following that, the three data sets used in the experiments are introduced in \ref{sec:datasets}
and finally the perturbation and reconstruction models are depicted in \ref{sec:reconstruction_models}.
Section \ref{sec:evaluation} deals with evaluation metrics.

\section{neural network}
\label{sec:nn_def}

In a classification setting, a \textbf{neural network} $\Phi$ is a function that maps an input $\vec x \in \R^d$ to an output $\in \R^C$,
often referred to as the unnormalized logits. 
The network makes its prediction by returning the index of the maximal value of the logits.
The classic neural network is a \textbf{multi-layer perceptron} (MLP) or fully-connected (FC) neural network.
Here, every input of a layer is connected to every output of the layer.
Every node of any layer receives its input as a weighted sum over all of the previous layer's activations.
This weighted sum is passed through a non-linear activation function, often one of either the rectified-linear unit (ReLU) or sigmoid, to form the node's activation. 
This is done for all nodes in a layer before being passed on to the next.
A full network thus is a composition of functions $\Phi_\ell$, or \textbf{layers} that 
successively act on the outputs $\vec h$ of the previous layer, called \textbf{activations} or \textbf{hidden states}. 

In a plain \textbf{fully-connected} neural network, the layers are each made up of an affine linear transformation and a non-linear activation function.

Formally, a fully-connected neural network of layer depth L is described as:

\[
    \Phi : \R^d \to \R^C
\]
\[
    \Phi = \Phi_\text{L} \circ \ldots \circ \Phi_1
\]
\[
    \vec h_\ell = (\Phi_\ell \circ \ldots \circ \Phi_1) (\vec x) =
    \Phi_\ell(\vec h_{\ell-1}) \in \R^{d_\ell}
\]
$\Phi_\ell$ then, is a mapping from $\R^{d_{\ell-1}}$ to $\R^{d_\ell}$ and $\R^{\textnormal L} = \R^C$.
% The space of a hidden state $\vec h$ is called a \textbf{feature-space}.
The input $\vec x$ is sometimes also referred to as $\vec h_0$ and $\R^{d_0} := \R^d$.
The layers themselves contain as parameters a \textbf{weight matrix} $\vec W$ and a \textbf{bias} $\vec b$
to form an affine linear transformation.
\[
    \Phi_\ell(\vec h) = \sigma (\vec W \vec h + \vec b) \comma{where}
    \vec W \in \R^{d_\ell \times d_{\ell-1}} \comma{and}
    \vec b \in \R^{d_\ell}
\]

The non-linear activation function $\sigma$ is often a function $\sigma:\R\to\R$ that is applied element-wise.



\section{Methods}
\label{sec:methods}

The following methods introduce various feature-mappings.
Each feature-mapping $\varphi$ induces a loss function $\loss_\varphi$ as outlined in \ref{eqn:statistics_loss}.
For every method there exists a class-dependent loss formulation $\loss_\varphi^{\mathcal C}$,
as given in \ref{eqn:class_statistics_loss}.
The abbreviations to the methods are given in the title in parenthesis. 
When referring to the class-dependent formulation \textit{CC} (class-conditional or
class-constrained) is added to the abbreviation.

The complete loss formulation that is used in the optimization process 
for a given feature-map $\varphi$, a target data set $\set A$ and a source data set $\set B$ 
is a composite loss given by
\[
    r_\text{stats}\loss_\varphi(\set A, \set B) + r_\text{crit}\loss_\text{crit}(\set B, y_{\set B}) \,,
\]
where $y_{\set B}$ are the target labels and $\loss_\text{crit}$ is the criterion loss, in this case, the cross-entropy loss.
It is the loss function that was used for optimizing the parameters of the neural network.



\subsection{Neural network - second-to-last layer (NN)}

Given a neural network $\Phi$ of depth L,
a feature-mapping can be obtained by mapping inputs to the second-to-last layer $\Phi_{\text{L}-1}$, 
the layer before the logits layer.
\[
    \varphi : \R^d \to \R^{d_\textnormal{L-1}} = (\Phi_\textnormal{L-1} \circ \dots \circ \Phi_1)
\]
Generally, this layer is considered to hold a highly abstracted representation of the data
and typically contains the signal in a comparatively high dimensional space, before it's
reduction to a $C$-dimensional vector by the logits-layer.
% which contains the least amount of information  on the input.
In convolutional neural networks, where one often deals with images
of a specified input shape (the Cartesian product of channels$\times$height$\times$width), 
the spatial relationship of features is kept throughout the forward propagation by the layers.
In this layer, the entire signal is flattened down to a single $d^{\text{L}-1}$-dimensional vector, 
before being passed to the logits-layer.
As normally no activation function is needed on the last layer when returning the index of the maximal value, 
the logits-layer has to be able to linearly classify its input - the last hidden state - in order to 
give the final prediction.
This leads one to believe that this layer contains the most "disentangled" representation of the data.


\subsection{Neural network - all layers (NN ALL)}

All hidden states together of the neural network can be seen as outputs of a collection of feature-maps [$\varphi_0, \dots, \varphi_\text{L}$], where
\begin{alignat*}{2}
    \varphi_\ell &: \R^d \to \R^{d_\ell} &&= (\Phi_\ell \circ \dots \Phi_1)
    \comma{for $\ell = 1, \dots$, L - 1}  \,, \\
    \varphi_\textnormal{L} &: \R^d \to \R^d &&= \Id \,.
\end{alignat*}





\subsection{Random projections (RP)}
To contrast the efficacy of the previous two feature-representations,
a feature-mapping involving one linear layer will be used, by selecting a number of  random linear projections.
A random projection is a linear mapping $r: \R^d \to \R$
It is created by choosing a normalized random vector $\vec v \in S^{d-1} = \{\vec x \in \R^d : \|\vec x\| = 1\}$.
It can be seen as  linear projection on to the one-dimensional subspace defined by the vector $\vec v \in \R^d$.
\[
    r(\vec x) = \vec v ^\top \vec x
\]
By choosing a total of $s_\text{RP}$ random vectors $\{v_1, \dots, v_{s_\text{RP}}\}$ one obtains a linear mapping $R: \R^d \to \R^n$:
\[
    R(\vec x) = \vec V \vec x =
    \begin{bmatrix}
        - \vec v_1 ^\top - \\
        \vdots \\
        - \vec v_n ^\top - \\
    \end{bmatrix}
    \vec x =
    \begin{bmatrix}
        \vec v_1 ^\top \vec x \\
        \vdots \\
        \vec v_n ^\top \vec x \\
    \end{bmatrix}
\]
%
% The loss function $\loss_R$ stays as was defined before.
In order to obtain more balance the outputs,
the projection can be centered around an origin $\vec o$.
% 
\[
    R_{\vec o} (\vec x) = \vec V (\vec x - \vec o) \,,
\]
where $\vec o \in \R^d$ is the new origin of the projection. 
The idea is to set the origin to be center of the target data set $\set A$. $\vec o = \mean (\set A)$
This way, the data after the projection will be centered around $\vec 0$.
Practically, this doesn't change much as the data is often normalized and $\vec 0$-centered beforehand,
yet the class-dependent variant can be modified to select an origin $\vec o_c$ for each class $c$.
Equation \ref{eqn:class_statistics_loss} then becomes:
% 
\begin{equation*}
    \loss _{\text{RP}} ^{\mathcal C} (\set A, \set B) =
    \sum _{\substack{c = 1 \dots C \\ \set A|_c,\, \set B|_c \text{ contain} \\ \text{at least 2 samples}}}
    \begin{alignedat}[t]{1}
        \|\mean ({R_{\vec o_c} (\set A|_c)}) - \mean ({R_{\vec o_c} (\set B|_c)}) &\| \\
        {} + \|\var ({R_{\vec o_c} (\set A|_c)}) - \var ({R_{\vec o_c} (\set B|_c)}) &\|  \,,
    \end{alignedat}
\end{equation*}
%
where $\vec o_c$ is set to the mean of each class $\mean ({\set A|_c})$.


\subsection{Random projections ReLU (RP ReLU)}
To further explore the importance of the non-linear activation functions contained within the network,
the previous method is modified by applying an activation function, in this case the ReLU to the output.
% 
\[
    R_{\vec o}^+ (\vec x) = (\vec V (\vec x - \vec o))^+ \,,
\]
where $(\,\cdot\,)^+ :\R^n \to \R^n$ is the projection onto the positive orthant. It applies $\max(0, \cdot)$ element-wise.

Since the layers of neural networks have a bias parameter that shifts the threshold of 
where an input can pass, this will also be incorporated.
\[
    R_{\vec o}^{\vec b} (\vec x) = (\vec V (\vec x - \vec o) + \vec b)^+ \,,
\]
where $\vec b \in \R^d$ is the bias. For a target dataset $\set A$, it is chosen as $\vec b \sim \mathcal N(\vec 0, \textnormal{diag}(\boldsymbol \sigma ^2))$, 
where $\boldsymbol \sigma ^2 = \var ({R_{\vec o}(\set A)})$.

The class-dependent variant can again make use for more suited biases $\vec b_c \sim \mathcal N( \vec 0, \textnormal{diag}(\boldsymbol \sigma _c^2))$, 
where $\boldsymbol \sigma_c ^2 = \var ({R_{\vec o_c}(\set A|_c)})$.
In this case, equation \ref{eqn:class_statistics_loss} becomes:
% 
\begin{equation*}
    \loss _{R^+} ^{\mathcal C} (\set A, \set B) =
    \sum _{\substack{c = 1 \dots C \\ \set A|_c,\, \set B|_c \text{ contain} \\ \text{at least 2 samples}}}
    \begin{alignedat}[t]{1}
        \|\mean ({R_{\vec o_c}^{\vec b_c} (\set A|_c)}) - \mean ({R_{\vec o_c}^{\vec b_c} (\set B|_c)}) &\| \\
        {} + \|\var ({R_{\vec o_c}^{\vec b_c} (\set A|_c)}) - \var ({R_{\vec o_c}^{\vec b_c} (\set B|_c)}) &\| 
    \end{alignedat}
\end{equation*}

\subsection{Randomly initialized neural network (RANDOM NN)}
To further study the importance of an optimized feature-representation of a trained neural network, 
the same neural network model with randomly initialized parameters will be evaluated and compared.
This could give insight to the importance of compositions of non-linear transformations

\subsection{Combinations (COMBINED)}
All previously defined feature-maps can be combined into a new collection of feature-maps,
in order to obtain a new loss formulation.
One combination in particular, the combination of all neural network layers (NN ALL) and random projections (RP) will be further examined in the following experiments.


\section{Data Sets}
\label{sec:datasets}

\subsection{GMM}
\label{sec:datasetgmm}
A dataset made up of Gaussian mixture models (in the following \textit{GMM}), 
is made up of $C$ classes, each of which contains $n_\text{mode}$ clusters or modes of multivariate Gaussian normal distributions. The probability-density function (pdf) is as follows.
\begin{align}
\label{eqn:gmm_distr}
    p(\vec x) &= \frac 1 C \sum_{c=1}^C p(\, \vec x \mid c \,) \nonumber\\
    p(\, \vec x \mid c \,) &= \frac 1 {n_\text{mode}} \sum _{m=1}^{n_\text{mode}}
    \mathcal N (\gamma \vec m_c + \lambda \boldsymbol \mu_c^{(m)}, \boldsymbol \Sigma_c^{(m)}) \, ,
\end{align}
where \\
% $\boldsymbole \theta = (
\XXX{upper indices for c?}
$\vec m_c \sim \mathcal N (\vec 0, \vec I)$ is the center for each class $c=1,\ldots, C$ ,\\
$\boldsymbol \mu_c^{(m)} \sim \mathcal N (\vec 0, \vec I)$ and
$\vec \Sigma_c^{(m)}$ are the center and the covariance matrix of the multivariate normal distribution
for each mode $m=1,\ldots, n_{\textnormal{mode}}$ and class $c=1,\ldots, C$.  \\
A positive semi-definite matrix $\vec \Sigma_c^{(m)}$ is generated by choosing $d$ eigenvalues $\vec e = (e_1, \ldots, e_d)$, $e_i \sim \mathcal U(\alpha, \beta)$, for all $i=1, \ldots, d$, for some $\alpha, \beta > 0$ and 
by sampling a random orthogonal matrix $\vec Q$. 
This can be done for example by choosing $\vec Q = \{q_{ij}\}_{i,j=0}^{d}$, 
where $q_{ij} \sim \mathcal N(0, 1)$
and creating an orthonormal basis via Gram-Schmidt. 
\begin{align*}
    \vec \Sigma_c^{(m)} = \vec Q^\top \text{diag}(\vec e) \vec Q
\end{align*}

\XXX{plot example}
For a given data set size, the labels are generated by equally sampling from all classes $1, \ldots, C$.
Then the input will be sampled according to the conditional probability-density function given by \eqnref{eqn:gmm_distr}.
The specific parameters used for generating the data set in all the experiments
are given in appendix \ref{AppendixParameters}.


\subsection{MNIST}
MNIST, a common data set used in pattern recognition, is made up of 70000 black-and-white images 
of handwritten digits from 0 to 9. 
Each image is made up of 28$\times$28 pixels, totaling an input dimension of $d=784$.
\XXX{show example, give source}

\subsection{CIFAR-10}
CIFAR10 contains 60000 colored images from 10 non-overlapping categories or classes.
Each image has an input shape of $(3, 32, 32)$ as it comes with 3 color channels 
and has a dimension of 32$\times$32 pixels, resulting in a total of $d=3072$ dimensions.
\XXX{show example, give source}





\section{Perturbation and Reconstruction Models}
\label{sec:reconstruction_models}


\subsection{Gaussian Mixture Models}

The \textbf{perturbation} for the GMM data set is modeled by
a random affine-linear transformation.
The linear part consists of an identity transformation with added Gaussian noise
and the translation vector is also sampled by Gaussian noise.
The standard deviation of the noise is controlled by a parameter $\kappa > 0$.
\[
    \delta(\vec x) = (\vec I + \kappa \vec N)\vec x + \kappa \mu \,,
\]
where $\vec N = \{n_{i, j}\}_{i j = 1}^{d}$, $n_{ij} \sim \mathcal N (0, 1)$ for all $i, j = 1 , \ldots, d$
and 
$\mu = \{\mu_i\}_{i=0}^d$, $\mu_i \sim \mathcal N(0, 1)$ for all $i=1,\ldots,d$.
This transformation is in invertible if $\det (\vec I + \kappa \vec N) \neq 0$ with P=1,
\XXX{talk about invertibility?}
For $\kappa = 0$, the identity transformation is obtained.

The \textbf{reconstruction model} for this data set is also an affine-linear transformation. 
$\rho(\vec x) = \vec A \vec x + \vec b$.
$\vec A$ and $\vec b$ make up the models learnable parameters and
these are first initialized to form the identity transformation ($\vec A = \vec I$, $\vec b = \vec 0$).
An optimal solution is given by
$\vec A^* = (\vec I + \kappa \vec N)^{-1}$, $\vec b^* = -\kappa \vec A^* \boldsymbol \mu$.
Proof:
\begin{equation}
\label{eqn:gmm_optimal}
\begin{split}
    \delta ( \rho^* (\vec x)) &= (\vec I + \kappa \vec N)(\vec A^* \vec x + \vec b^*) + \kappa \boldsymbol \mu \\
    &= (\vec I + \kappa \vec N)\vec A^* (\vec x  - \kappa \boldsymbol \mu) + \kappa \boldsymbol \mu \\
    &= \vec x  -\kappa \boldsymbol \mu + \kappa \boldsymbol \mu \\
    &= \vec x
\end{split}
\end{equation}

\subsection{Image Data Sets}

\subsubsection{Perturbation model}

The \textbf{perturbation model} for the image data sets MNIST and CIFAR10
is a composition of additive Gaussian noise and two noise-controlled 2d-convolutions.
\XXX{talk about 2d-convolutions?}

\[
    \delta(\vec x) = \vec K_2(\vec K_1(\vec x + \mu)
\]

The convolutions $\vec K_n$ are given by kernel matrices $\vec k_n$ of size $3\times3$ for $n=1,2$.
The kernel matrices are set to the identity kernel with added Gaussian noise, controlled by $\kappa$.
The identity kernel for the MNIST data set (which only contains one "color" channel) is given by 
% \eqnref{eqn:kernelid}.
\begin{equation*}
% \label{eqn:kernelid}
    \vec k_{\text{id}} = \begin{bmatrix}
        0 & 0 & 0 \\
        0 & 1 & 0 \\
        0 & 0 & 0 \\
    \end{bmatrix} \,.
\end{equation*}
For the CIFAR10 data set, where 3 color channels are present, the input shape is of (3, 32, 32). 
In order to obtain the same output shape, 
a total of $3*3=9$ convolution kernels $\vec k^{[i,j]}$ for $i,j=1,2,3$ are needed per convolution.
The output of channel $j$ is given by $\sum_{i=1}^{3} \vec x * \vec k^{[i, j]}$.
The identity convolution can be achieved by setting 
\[
    \vec k^{[i,j]} = \begin{cases}
        \vec k_{\text{id}} &, \text{ if } i = j\\
        \vec 0 &, \text{ otherwise} \\
    \end{cases} \,.
\]

This convolution operator is in general not invertible for $\kappa > 0$.
\XXX{Proof??}
Again, for $\kappa = 0$, the identity transformation is obtained.

For the output to contain the same dimensions as the input, the input needs to be \textit{padded}.
The 'reflection' padding was chosen.
\XXX{Should elaborate}


\subsubsection{Reconstruction model}

\begin{figure}
\begin{minipage}{0.5\textwidth}
\centering
\input{Figures/ChartInvertblock}
% \caption*{ResNet - Residual Layer}
\end{minipage}
\begin{minipage}{0.5\textwidth}
\centering
\begin{tikzpicture}[node distance=0.58cm and 1.7cm, auto]

\node [] (input) {Input};
\node [layer, below= 1cm of input, fill=Dandelion!9] (conv1) {Conv1x1};
\node [layer, below= of conv1] (res1) {Residual Layer};
\node [layer, below= of res1] (res2) {Residual Layer};
\node [below= of res2, minimum height=0.8cm] (res3) {\dots};
\node [layer, below= of res3] (res4) {Residual Layer};
% \node [layer, below= 1cm of input, fill=Dandelion!9] (conv1) {Conv3x3};
% \node [layer, below= of conv1, fill=Blue!5] (bn1) {BatchNorm};
% \node [layer, below= of bn1, fill=Mahogany!7] (relu) {ReLU};
% \node [layer, below= of relu, fill=Dandelion!9] (conv2) {Conv3x3};
% \node [layer, below= of conv2, fill=Blue!5] (bn2) {BatchNorm};
\node [below= 1cm of res4] (output) {Output};

% \path (input) -- node[anchor=center] (branch) {} (conv1);
% \path (bn2) -- node[circle, draw, minimum size=0.6cm, anchor=center] (plus) {} (output);
% \node at (plus) {+};

% \node [right= 1.6cm of branch] (dummy) {};
\node [draw, dashed, inner sep=0.25cm, fit=(res1)] {};

% \draw [larrow, rect connect v=2cm] (branch.center) to (plus.east);

\draw [larrow] (input) to (conv1);
\draw [larrow] (conv1) to (res1);
\draw [larrow] (res1) to (res2);
\draw [larrow] (res2) to (res3);
\draw [larrow] (res3) to (res4);
\draw [larrow] (res4) to (output);
% \draw [larrow] (bn1) to (relu);
% \draw [larrow] (relu) to (conv2);
% \draw [larrow] (conv2) to (bn2);
% \draw [larrow] (bn2) to (plus);
% \draw [larrow] (plus) to (output);


\end{tikzpicture}
% \caption*{ResNet - Residual Layer}
\end{minipage}
\caption{ResNet architecture}
\label{fig:resnet}
\XXX{add ReLU out + 1x1conv}
\end{figure}

The model for the reconstruction task on images is given by a residual network architecture
as outlined in \ref{fig:resnet}. 
The network \textit{width} - the number of individual convolutional kernels - is controlled by the parameter $s_\text{width}$, the \textit{depth} - the number of residual layers - by the parameter
$s_\text{depth}$.

The default parameters for the experiments are given in \ref{AppendixParameters}.




\section{Evaluation}
\label{sec:evaluation}

\begin{tikzpicture}[node distance=1cm and 1.7cm, auto]

\node (A) [node, fill=red!7] {\textbf{Target} \\ $\set A$};
\node (B) [node, below= of A] {\textbf{original} \\ $\set B_\text{true}$};
\node (C) [node, below= 2.2cm of B, fill=blue!0] {\textbf{Validation} \\ $\set C_{true}$};

\node (Bp) [node, right= of B, fill=blue!2] {perturbed \\ \textbf{Source} \\ $\set B$};
\node (Br) [node, right= of Bp, fill=blue!7] {\textbf{reconstructed} \\ $\rho^* (\set B)$};


\draw [arrow, dashed, thin] (B) -- (Bp) node (Pert) [midway, function, dashed] {$\delta$};
\draw [arrow] (Bp) -- (Br) node (Rec) [midway, function] {$\rho^*$};
\node [fill=white, below= 0cm of Pert] {\footnotesize \textit{unknown}};
\node [fill=white, below= 0cm of Rec] {\footnotesize \textit{learned}};


\node (Net) [function, fill=gray!5, above= of Br] {Neural \\ Network};
\node (vNet) [function, fill=gray!5, right= of Net] {verification \\ Neural \\ Network};

\draw [arrow, Mahogany, thin, bend left=20] (A) to (Net);
\draw [arrow, Mahogany, thin, bend left=20] (A) to node[below] {\footnotesize trained} (vNet);
\draw [arrow, Blue, thin] (Net.210) to[bend right=20] node [midway, above, sloped] {\footnotesize optimized} (Rec.north);

\node (Cp) [node, right= of C, fill=blue!0] {\textbf{perturbed} \\ $ {\set C}$};
\node (Cr) [node, right= of Cp, fill=blue!0] {\textbf{reconstructed} \\ $\rho ^*( {\set C})$};


\draw [arrow, dashed, thin] (C) -- (Cp);
\draw [arrow] (Cp) -- (Cr);

\draw [arrow, dashed, thin] (C) -- (Cp) node (Pert) [midway, function, dashed] {$\delta$};
\draw [arrow] (Cp) -- (Cr) node (Rec) [midway, function] {$\rho^*$};

\draw [arrow, <->, thin, OliveGreen] (Br.south) to [rect connect h=-0.75cm] (B.south);
\draw [arrow, <->, thin, OliveGreen] (Cr.north) to [rect connect h=0.75cm] (C.north);

\path (Br.east) to [out=0, in=270]  node [near end, OliveGreen, xshift=0.1cm] {accuracy} (vNet.265);
\draw [arrow, thin, OliveGreen] (Br.east) to [out=0, in=320] (Net.315);
\draw [arrow, thin, OliveGreen] (Cr.east) to [out=0, in=320] (Net.325);
\draw [arrow, thin, OliveGreen] (Cr.east) to [out=0, in=270] (vNet.south);


\node [OliveGreen] at ($(Bp)!0.5!(Cp)$) {IQA metrics};

\node [below= 1.5cm of Pert] (Id) {Id};
\path (Id) -| node[anchor=center] (Idr) {$\hat {\Id}$} (Rec);
\draw [arrow] (Id) -- 
node[near start, xshift=0.2cm, function] {$\delta$} 
node[near end, xshift=-0.2cm, function] {$\rho^*$} (Idr);
\draw [arrow, <->, thin, OliveGreen] (Idr.south) to [rect connect h=-0.6cm] (Id.south);
\node [OliveGreen, yshift=-1.2cm] at ($(Id)!0.5!(Idr)$) {rel. error};
% \node [circle, fill=blue] at (Idr){};
% \draw (0,0)|-node{mid}(2,3);
% \node [below= 1cm of Rec] {};
% \node (Cbelow) [below= of Cp] {EEE};
% \draw [arrow, dashed, thin] (C) -- (Cp) node (Pert) [midway, function, dashed] {$\delta$};
% \draw [arrow] (Cp) -- (Cr) node (Rec) [midway, function] {$\rho^*$};

\end{tikzpicture}

The result of the reconstruction task on the GMM data set can be evaluated by the \textbf{accuracy}
of the neural network. This however, is likely to exhibit a strong bias in the classification accuracy 
towards the data set, since, the data set $\set B$ and, in many of the presented methods,
the neural network itself were used in the optimization process.
For this reason, a validation set $\set C$ will undergo the same transformations 
as $\set B$ and the correct classification rate will be reported as the \textbf{validation accuracy}. 
Since none of the data points in $\set C$
were used during the optimization, it is, to some extent a measure of generalization
of the learned transformation $\rho^*$.

A second measure of generalization can be made by evaluating the accuracy of an
independently trained neural network, the verifier neural network.
The accuracy of the validation set $\set C$ on this second neural network 
constitutes the \textbf{verifier accuracy}.

For the GMM data set, since the explicit probability density function (pdf) is known, 
(see \ref{eqn:gmm_distr}), the estimated \textbf{cross-entropy}, given by
\begin{equation}
\label{eqn:cross_entropy}
    H(\set A) = - \frac 1 {|\set A|} \sum_{i=1}^{|\set A|} \log p(\vec x_i)
\end{equation}
is used as a measure of reconstruction quality of $\rho^*$.

For the main reconstruction task, 
five metrics will be studied to determine a methods success (along with visual appeal).
For one, the accuracy of the reconstructed data set will be measured by the neural network.
The relative l2-error, the peak signal-to-noise ratio, the accuracy of the neural network and the accuracy on a verifier network.
Alongside, a validation data set $\set C$ will measure generalization of the found reconstruction $\rho$ to a data set to which it was not optimized for.
Since the original neural network was also used in the optimization process, a verifier network $\Phi_{\text{ver}}$ will be used as a second evaluation of the accuracy.

If the perturbation $\delta$ is known, then one can calculate the \textbf{relative error} of the identity vectors. 
\begin{equation}
\label{eqn:fro_error}
    \varepsilon_F = \frac {\|\widehat {\vec I_d} - \vec I_d\|_F} {\|\vec I_d\|_F} \,,
\end{equation}
where $\widehat {\vec I_d} = \begin{pmatrix} \rho (\delta (\vec e_1), \dots, \rho (\delta (\vec e_d) \end{pmatrix}$ and $\vec e_i$ is the i-th unit vector of the standard basis.

Then \eqnref{eqn:fro_error} becomes
\begin{equation}
\label{eqn:l2_error}
    \varepsilon_F = \frac 1 {\sqrt d} \sqrt{ \sum_{i=0}^d \|\rho (\delta (\vec e_i)) - \vec e_i)\|_2^2}
\end{equation}

For the GMM data set, where the distortion model $\delta(\vec x) = (\vec I + \kappa \vec N) \vec x + \boldsymbol \mu$
and reconstruction model $\rho(\vec x) = \vec A \vec x + \vec b$ are both affine-linear transformations,
this is motivated as follows.
\begin{align*}
    \|\rho(\delta (\vec x)) - \vec x \| 
    &= \|(\vec I + \kappa \vec N) (\vec A \vec x + \vec b) + \kappa \boldsymbol \mu  - x \| \\
    &= \|(\vec I + \kappa \vec N) (\vec A \vec x - (\vec I + \kappa \vec N)^{-1} \vec x) 
        + (\vec I + \kappa \vec N) \vec b + \kappa \boldsymbol \mu  \| \\
    &= \|(\vec I + \kappa \vec N) (\vec A - (\vec I + \kappa \vec N)^{-1}) \vec x 
        + (\vec I + \kappa \vec N) (\vec b + (\vec I + \kappa \vec N)^{-1} \kappa \boldsymbol \mu)  \| \\
    &\leq \|\vec I + \kappa \vec N\| \left(\|(\vec A - (\vec I + \kappa \vec N)^{-1})\| \|\vec x\| 
        + \|(\vec b + (\vec I + \kappa \vec N)^{-1} \kappa \boldsymbol \mu)  \| \right)\\
    &= \|\vec I + \kappa \vec N\| \left(\|\vec A - \vec A^* \| \|\vec x\| 
        + \|\vec b - \vec b ^*\| \right) \,,
\end{align*}
where $\vec A ^*$ and $\vec b^*$ are the parameters to the exact inverse transformation of $\delta$ 
(see \ref{eqn:gmm_optimal}).
For a unit vector $\vec e_i$ this becomes:
\begin{equation*}
    \|\rho(\delta (\vec e_i)) - \vec e_i \| 
    \leq \|\vec I + \kappa \vec N\| \left(\|\vec A - \vec A^* \| 
        + \|\vec b - \vec b ^*\| \right) \,.
\end{equation*}
\eqnref{eqn:fro_error} then, is bound by
\begin{align*}
    \varepsilon_F 
    &\leq \|(\vec I + \kappa \vec N)\|
    \sqrt{ \frac 1 d \sum_{i=0}^d \left(\|\vec A - \vec A^* \|  + \|(\vec b - \vec b ^*)  \| \right)^2} \\
    &= \|(\vec I + \kappa \vec N)\| \left(\|\vec A - \vec A^* \|  + \|(\vec b - \vec b ^*)  \| \right) \,.
\end{align*}

These calculations are no longer valid for the image data sets, as it is not guaranteed that 
an inverse transformation exists, given the convolutional distortion models. The relative
error, as defined in \ref{eqn:l2_error} will be reported as a metric, but it is important to
note that the reconstruction model will behave differently on a unit-vector of the standard-basis
than on an image from the data set, due to its non-linear nature.

A metric that is commonly encountered in image quality assessment (IQA), is the 
\textbf{peak signal-to-noise ratio} (\textbf{PSNR}).
It is formed between two images $\vec x$ and $\vec y$ and is calculated as follows.
\[
    PSNR(\vec x, \hat {\vec x}) = 20 \log_{10} \left (\frac {\max_i(\vec x_i)} {\|\vec x-\hat {\vec x}\|_2} \right )
\]
It is a common measure of image quality when assessing image compression algorithms and is measured in decibels db.
It is similar to the relative error in that it uses the mean-squared error, 
though it assessed on data points directly and therefore could be more indicative of actual 
reconstruction quality, whereas the l2-error can be seen as a metric describing the general invertibility.
For a batch of images, the score is averaged over the images to give an estimated mean PSNR score.

Both, the l2-error and the PSNR compare individual values in pixels of images for their calculations.
Because of this, a low score in both of these metrics might not necessarily mean that a reconstruction
model has a bad performance. A near optimal reconstruction function can suffer from a heavy penalty
in these scores due to a slight shift in translation of pixel values that humans would not even be
able to notice.
These translations can easily occur in convolutional models.
For this reason, one last metric the, \textbf{structural similarity index measure} (\textbf{SSIM})
is considered. Since it does not compare individual pixel values, it is more robust to 
image translations.
It is given as follows.
\[
    \text{SSIM}(\vec x, \vec y) 
    = \frac {(2 \mu_{\vec x} \mu_{\vec y} + c_1)(2\sigma_{\vec x\vec y} + c_2)}
    {(\mu_{\vec x}^2 + \mu_{\vec y}^2 + c_1)(\sigma_{\vec x}^2 + \sigma_{\vec y}^2 + c_2)}
\]
\XXX{explain}


\subsection{Results}
